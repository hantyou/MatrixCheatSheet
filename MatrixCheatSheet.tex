\documentclass[onecolumn]{IEEEtran}
\IEEEoverridecommandlockouts
% The preceding line is only needed to identify funding in the first footnote. If that is unneeded, please comment it out.
\usepackage[noadjust]{cite}
\renewcommand\citepunct{, }
%\usepackage[caption=false]{subfig}
\usepackage{subcaption}
\usepackage{amsmath,amssymb,amsfonts}
\usepackage[ruled,vlined,linesnumbered]{algorithm2e}
\usepackage{bm}
\usepackage{graphicx}
\usepackage{enumitem}
\usepackage{booktabs}
\usepackage{textcomp}
\usepackage{color,soul}
\usepackage{multirow}
\usepackage{comment}
\usepackage{tcolorbox}
\usepackage{balance}
\usepackage{tabularx}
\usepackage[yyyymmdd,hhmmss]{datetime}
\usepackage{array}
\usepackage{textcase}
\usepackage[tablename=TABLE]{caption}
\usepackage[colorlinks=true, allcolors=blue]{hyperref}
\def\BibTeX{{\rm B\kern-.05em{\sc i\kern-.025em b}\kern-.08em
		T\kern-.1667em\lower.7ex\hbox{E}\kern-.125emX}}
%\usepackage[usenames,dvipsnames]{xcolor}
\usepackage{listings}
\lstset{
	%backgroundcolor=\color{backcolour},   
	commentstyle=\color{codegreen},
	keywordstyle=\color{blue},
	numberstyle=\tiny\color{codegray},
	stringstyle=\color{codepurple},
	basicstyle=\footnotesize,
	escapeinside=``,
	breaklines,
	columns=fullflexible,
	identifierstyle={},
	stringstyle=\ttfamily,
	extendedchars=false,
	linewidth=1\textwidth,
	numbers=left,
	numbersep=2 em, 
	tabsize=4,
	frame=leftline,
	showstringspaces=false
}

\newcommand{\lrangle}[1]{\langle#1\rangle}
\newcommand{\lrvert}[1]{\left\vert#1\right\vert}
\newcommand{\lrVert}[1]{\left\Vert#1\right\Vert}
\usepackage{tikz}
\usepackage{tikz-qtree}
\usepackage{style/algorithm2estyle}
\usetikzlibrary{trees}
\IEEEoverridecommandlockouts

\SetKwInput{KwData}{Input}
\SetKwInput{KwResult}{Output}
\SetKwInOut{Init}{Init}
\newtheorem{theorem}{Theorem}
\begin{document}
	\title{Matrix and vector derivative cheat sheet}
	\author{
		\IEEEauthorblockN{Peiyuan Zhai}
		%\IEEEauthorblockA{5276179}
	}
	\maketitle
 \begin{enumerate}
 	\item \[\nabla_{\mathbf{x}}\mathbf{Ax}=\left[\begin{matrix}
 		\nabla_{\mathbf{x}}\tilde{\mathbf{a}}_1\mathbf{x}\\\nabla_{\mathbf{x}}\tilde{\mathbf{a}}_2\mathbf{x}\\\vdots\\\nabla_{\mathbf{x}}\tilde{\mathbf{a}}_m\mathbf{x}
 	\end{matrix}\right]=[\tilde{\mathbf{a}}_1^T,\tilde{\mathbf{a}}_2^T,\cdots,\tilde{\mathbf{a}}_m^T]=\mathbf{A}^T\]
 \item $\mathbf{x}^T\mathbf{Ax}=\sum\limits_{i=1}^nx_i\tilde{\mathbf{a}}_i\mathbf{x}$\\
 $\Rightarrow$ $\frac{\partial \mathbf{x}^T\mathbf{Ax}}{\mathbf{x}_l}=\sum_{i=1}^n\mathbf{x}_i\tilde{a}_{i,l}+\tilde{\mathbf{a}}_l\mathbf{x}=\mathbf{a}_l^T\mathbf{x}+\tilde{\mathbf{a}_l}\mathbf{x}$\\
 $\Rightarrow$ $\frac{\partial\mathbf{x}^T\mathbf{Ax}}{\partial \mathbf{x}}=\mathbf{A}^T\mathbf{x}+\mathbf{Ax}$
 \item (Derivative in a trace) $\mathbf{A}\in\mathbb{R}^{n\times n}$
 \[\boxed{\frac{tr(\mathbf{A}d\mathbf{X})}{d\mathbf{X}}=\mathbf{A}^T}\]
 \item $\mathbf{A}\in\mathbb{R}^{n\times m}$ and $\mathbf{B}\in\mathbb{R}^{m\times n}$
 \[tr(\mathbf{AB})=\sum\limits_{i=1}^ma_{1i}b_{i1}+\cdots+\sum\limits_{i=1}^ma_{ni}b_{in}\]\\
 $\Rightarrow$
 \[\boxed{\nabla_{\mathbf{A}}tr(\mathbf{AB})=\mathbf{B}^T}\]
 \item \begin{equation}
 	\begin{split}
 		\frac{\nabla f(\mathbf{Ax})}{\nabla x_i}&=\sum\limits_{k=1}^n\frac{\nabla f(\mathbf{Ax})}{\nabla(\mathbf{Ax})_k}\cdot\frac{\nabla(\mathbf{Ax})_k}{\nabla x_i}=\sum\limits_{k=1}^n\frac{\nabla f(\mathbf{Ax})}{\nabla(\mathbf{Ax})_k}\cdot\frac{\nabla(\tilde{\mathbf{a}}_k^T\mathbf{x})}{\nabla x_i}\\
 		&=\sum\limits_{k=1}^n\frac{\nabla f(\mathbf{Ax})}{\nabla(\mathbf{Ax})_k}\cdot a_{ki}=\sum\limits_{k=1}^n\nabla_{x_k}f(\mathbf{Ax})a_{ki}\\
 		&=\mathbf{a}_i^T\nabla f(\mathbf{Ax})
 	\end{split}
 \end{equation}
$\Rightarrow$  \fbox{$\nabla_{\mathbf{x}}f(\mathbf{Ax})=\mathbf{A}^T\nabla f(\mathbf{Ax})$}\\
Similarly, one can derive that 
\[\frac{\nabla^2f(\mathbf{Ax})}{\nabla x_i\nabla x_j} = \mathbf{a}_i^T\nabla^2f(\mathbf{Ax})\mathbf{a}_j \]
$\Rightarrow$  \fbox{$\nabla^2_{\mathbf{x}}f(\mathbf{Ax})=\mathbf{A}^T\nabla^2f(\mathbf{Ax})\mathbf{A}$}
\item \[\nabla_{\mathbf{A}}tr(\mathbf{AB}\mathbf{A}^T\mathbf{C})=\mathbf{CAB}+\mathbf{C}^T\mathbf{AB}^T\]
\item Matrix Inversion Lemma (MIL, Woodbury matrix identity)
\[(A + UCV^T)^{-1} = A^{-1} - A^{-1}U(C^{-1} + V^TA^{-1}U)^{-1}V^TA^{-1}\]
\begin{enumerate}
	\item Variant 1: $[\frac{1}{\sigma^2}\mathbf{A}^T\mathbf{A}+\boldsymbol{\Sigma}_{\boldsymbol{\alpha}}]^{-1} = \boldsymbol{\Sigma}_{\boldsymbol{\alpha}}^{-1}-\boldsymbol{\Sigma}_\alpha^{-1} \mathbf{A}^T\left(\frac{1}{\sigma^2} \mathbf{I}+\mathbf{A} \boldsymbol{\Sigma}_\alpha^{-1} \mathbf{A}^T\right)^{-1} \mathbf{A} \boldsymbol{\Sigma}_{\boldsymbol{\alpha}}^{-1}$
\end{enumerate}
 \end{enumerate}
	\clearpage
	% \bibliographystyle{ieeetr}
	% \bibliography{main}
\end{document}


